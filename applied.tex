\section{Document classification}
Task: given a set of labelled documents, construct a classifier to
decide for unlabeled documents.

Features
\begin{itemize}
\item Words of the documents
  \begin{itemize}
  \item bag of words
  \item document vector
  \end{itemize}
\item More detaild information on words
  \begin{itemize}
  \item Phrases
  \item Word fragments
  \item Grammatical features
  \end{itemize}
\item Any metadata about document and author
\end{itemize}

\subsection{Challenges}
The feature space is very high-dimensional
\begin{itemize}
\item Vocabulary size
\item All words bigrams
\item All character trigrams
\end{itemize}

Dealing with high-dimensionality
\begin{itemize}
\item Feature selection
\item Dimensionality reduction, e.g.\ word embedding
\item Classification algorithm that scale well
\end{itemize}

\subsection{Simple method: kNN}
use vector space model. \\
To classify D
\begin{itemize}
\item retrieve k nearest neighbors in training set according to vector
  space model
\item Choose majority class as the class of D
\end{itemize}

If k is large: $ \frac{\#C}{k} $ estimates $ P(C \mid D) $ the
probability that the document has indeed class C

\subsection{Naîve bayes classifier}
\begin{itemize}
\item Apply bayes law
\item Feature: bag of words model: all words and their count in the
  document
\item Training:
  \begin{itemize}
  \item How characteristic is word w for class C? \\
    $ P(w \mid C) = \frac{\mid w \in D, D \in C \mid +
      1}{\sum_{w'}\mid w' \in D, D \in C \mid + 1} $
  \item How frequent is class C \\
    $ P(C) = \frac{D \in C}{\mid \Delta \mid} $
  \end{itemize}
\end{itemize}

\subsubsection{Using bayes classifier}
Using bayes law the prob the class is C for D \\
$ P(C \mid D) \propto P(C) \prod_{w \in D} P(w \mid C) $ \\
Thus most probable class is\\
$ C_{NB} = argmax_{C} (log P(C) + \sum_{w \in D} log P(w \mid C)) $

\subsection{Classification using word embeddings}

Probability that word w occurs with context word c \\
$ P(D = 1 \mid w,c ; \theta) $

Consider instead of (word, context) pairs, (class, paragraph) pairs
\begin{itemize}
\item Word w $ \rightarrow $ Class label C
\item Learn $ P(C \mid p)= \frac{e^{p_v \cdot v_C}}{\sum_{C'} e^{-v_p
      \cdot v_{C'}}} $ (SGD)
\item Paragraph feature: $ v_p = \sum_{w \in p} v_w $
\item Then predict label from paragraph features
\end{itemize}

\subsection{Fasttext}
Classifier based on word embeddings \\
Extensions:
\begin{itemize}
\item Using word n-grams (phrases)
\item Subword information (character n-grams)
\item Possible to build vectors for unseed words!
\end{itemize}

\subsection{Document classfication summary}
Widely studied problem with many applications
\begin{itemize}
\item Spam filtering, sentiment analysis, document search
\item Increasing interest
  \begin{itemize}
  \item Powerful methods using very large corpuses
  \item Information filtering on the internet
  \end{itemize}
\end{itemize}

\section{Recommender systems}
\subsection{Definition}
Given a user model and a set of items, a recommender system is a
function that helps to mathc users with items by ranking the in order
of decreased relevance.

\subsection{Approaches}
\begin{itemize}
\item Collaborative: Tell what other people like
\item Content based: show me more of what I liked
\end{itemize}

\subsection{Collaborative filtering}
Widely used approach
\begin{itemize}
\item Used by large e-commerce sites
\item Applicable in many domains
\item Well understood
\end{itemize}

Wisdom of the crowd
\begin{itemize}
\item Users give ratings to items
\item Users with similar testes in the past will have similar test in
  the future
\end{itemize}

\subsection{User-based collaborative filtering}
Given a user U and an item I not rated by U, estimate the rating $
r_U(I) $ by
\begin{itemize}
\item Rind a set of users $ N_U $ who liked the same items as U in the
  past and who have rated I
\item Aggregate the ratings of I provided by $ N_U $
\end{itemize}

Compute ratings for all items not rated by U and keep the best-rated
ones.
\subsection{Similarity between users}
Pearson correlation coefficient: \\
$ sim(x, y) = \frac{\sum_{i = 1}^N (r_x(i) - \mathsf{r_x})(r_y(i) -
  \mathsf{r_y})}{\sqrt{\sum_{i = 1}^N (r_x(i) - \mathsf{r_x})^2}
  \sqrt{\sum_{i=1}^N (r_y(i) - \mathsf{r_y})^2}} $ \\

Cosine similarity: \\
$ sim(x, y) = cos(\theta) = \frac{\sum_{i = 1}^N r_x(i) \cdot r_y(i)
}{\sqrt{\sum_{i = 1}^N r_x(i)^2} \sqrt{\sum_{i=1}^N r_y(i)^2}} $

\begin{itemize}
\item x,y :users
\item N items i rated by both x and y
\item $ r_x(i) $: rating of user x of item i
\item $ \mathsf{r_x} $: average ratings of user x
\end{itemize}

\subsection{Ratings aggregation}
common function $ r_x(a) = \mathsf{r_x} + \frac{\sum_{y \in N_U(x)}
  sim(x, y)(r_y(a) - \mathsf{r_y})}{\sum_{y \in N_U(x)} \mid
  sim(x,y)} $
\begin{itemize}
\item $ N_U(x) $:neighbours of user x
\item $ a $: item not rated by x
\end{itemize}

The aggregation function calculates whether the ratings of neighbours
are higher or lower than average. We can consider this as the bias of
the user y w.r.t.\ item a. The bias is then added/subtracted from
user's x average rating.

\subsection{User-based collaborative filter}
Problems
\begin{itemize}
\item Cold start: users or items without ratings
\item Scalability: large numbers of users
\item Data dispersion: highly variable ratings
\end{itemize}

Possible solution
\begin{itemize}
\item Item-based collaborative filtering
\end{itemize}

\subsection{Item-based collaborative filtering}
Use similarity between items not users
\begin{itemize}
\item Given a user U and an item I not rated by U, estimate the rating
  $ r_U(I) $ by
  \begin{itemize}
  \item Find a set of items $ N_I $ who are similar to I and that have
    been rated
  \item Aggregate the ratings of the items in $ N_I $ and use
    aggregation as prediction of $ r_U(I) $
  \end{itemize}
\end{itemize}

\subsubsection{Aggregate the ratings}
$ r_x(a) = \frac{\sum_{b \in N_I(a)} sim(a, b) r_x(b)}{\sum_{b \in
    N_I(a)} \mid sim(a, b) \mid}$
\begin{itemize}
\item $ N_I(a) $: neighbours of item $ a $
\item $ b $: item rated by $ x $
\end{itemize}

\begin{itemize}
\item Does not solve the scalability problem but
\item Calculate all pair-wise item similarities in advance
\item Items similarities are more stable
\item The neighborhood $ N_I(a) $ to be considered at runtime is small
\end{itemize}

\subsection{Content-based recommandation}
Collaborative filtering needs only ratings, not information about
items. But item content can provide some useful information. \\
Given the items that have been rated by user U in the past
\begin{itemize}
\item Find the items that are similar to past items
\item Aggregate ratings of most similar items
\end{itemize}

Use method from information retrieval to extract content of an
item. \\
Given an item and a textual description, compute TF-IDF weights \\
$ w(t,a) = tf(t, a) \cdot idf(t) = \frac{freq(t,a)}{max_{s \in T}
  freq(s, a)} log(\frac{N}{n(t)})$
\begin{itemize}
\item N: number of items to recommend
\item n(t): number of items where term t appears
\end{itemize}

\subsubsection{Similarity between items}
Cosine similarity
$ sim(a,b) = cos(\theta) = \frac{\sum_{t = 1}^T w(t,a) \cdot
  w(t,b)}{\sqrt{\sum_{t = 1}^T w(t,a)^2} \sqrt{\sum_{t=1}^T w(t,b)^2}}
$
\begin{itemize}
\item a, b: items
\item T: terms t that appear in all items
\item w(t, a): TF-IDF of term t in item a
\end{itemize}

\subsubsection{Making predictions}

Given a set of items D that have been rated by the user, find nearest
neighbours in D of new item a. Use ratings of nearest neighbours to
predict rating with the aggregation function: \\
$ r_x(a) = \frac{\sum_{b \in N_I(a)} sim(a, b) r_x(b)}{\sum_{b \in
    N_I(a)} \mid sim(a, b) \mid} $

\subsection{Discussion of content-based}
Problems:
\begin{itemize}
\item Cold start problem
\item Content be limited or impossible to extract (non-text)
\item Tends to recommend more of the same
\end{itemize}

More scalable: TF-IDF of items can be computed offline


%%% Local Variables:
%%% mode: latex
%%% TeX-master: "master"
%%% End:
