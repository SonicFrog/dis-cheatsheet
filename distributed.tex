\section{Distributed retrieval}
\subsection{Retrieval processing}

\textbf{Centralized retrieval}
\begin{itemize}
\item Aggregate the weights for ALL documents by scanning the posting
  lists of the query terms
\item Scanning is relatively efficient
\item Computationally quite expensive (memory, processing)
\end{itemize}

\textbf{Distributed retrieval}
\begin{itemize}
\item Posting lists for different terms stored on different nodes
\item The transfer of complete postings lists can become prohibitively
  in terms of bandwith consumption
\end{itemize}
Is it necessary to transfer the complete posting list to identify the
top-k documents?

\subsection{Fagin's algorithm}
Entries in posting lists are sorted according to the tf-idf weights
\begin{itemize}
\item Scan in parrallel all lists in round-robin till k documents are
  detected that occurs in all lists
\item lookup the missing weights for documents that have not been seen
  in all lists
\item select the top-k elements
\end{itemize}
This probably returns the top-k documents.

\textbf{Discussion}
\begin{itemize}
\item $ O(\sqrt{k n}) $ entries are read in each list for n
  documents
\item Assuming that entries are uncorelated
\end{itemize}

In distributed settings optimizations to reduce the number of
roundtrips: send a longer prefix of one list to the other node
Useful for many applications:
\begin{itemize}
\item multimedia, image retrieval
\item Top-k processing in relational databases
\item Document filtering
\item Sensor data processing
\end{itemize}


%%% Local Variables:
%%% mode: latex
%%% TeX-master: "master"
%%% End:
