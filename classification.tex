\section{Classification}
\subsection{Classification problem}
\textbf{Input}: set of objects with attributes and one class label \\

\textbf{Output}: A model that returns the class label given the object
attributes. Model is a function represented as rules, decision trees,
\ldots Classification is \texttt{supervised} ML.

\subsection{General approach}
Model is learnt from a set of objects with known labels:
\textbf{training set}. Quality is evaluated by comparing the predicted
labels with known objects: \textbf{test set}. The two sets must be
independent or over-fitting occurs. Model then applied to data with
unknown label: \textbf{prediction}.

\subsection{Problem formulation}
Given a database D with n data items described by d attributes and one
categorical attributes (class label C). \\
Find a function f: $ X^d \rightarrow C $ rules, decision tree, formula \\
Such that it
\begin{itemize}
\item classifies \textit{accurately} the items in \textit{training
    set}
\item \textit{{generalises}} well for the items in \textit{test set}
\end{itemize}

\subsection{Characteristics}
Predictive accuracy \\
Speed and scalability
\begin{itemize}
\item Time to build model
\item Time to use model
\item In memory vs.\ on disk processing
\end{itemize}
Robustness: handling noise, outliers, missing values
Interpretability
\begin{itemize}
\item Understading the model and its decisions (whitebox) vs.\ white
  box
\item Compactness of model
\end{itemize}

\subsection{Decision tree induction}
Properties: \\
flow-chart like tree-structure
\begin{itemize}
\item Nodes are tests on a single-attribute
\item Branches are attributes values
\item Leaves are marked with class labels
\end{itemize}
Score function: classification accuracy. \\
Optimisation: top-down tree construction + pruning

\subsection{Decision tree algorithm}
Tree construction (top-down divide and conquer)
\begin{itemize}
\item All training samples belong to the root
\item Partitioned recursively on ``most discriminative'' attributes
\item Discriminative power based on information gain
\end{itemize}
Partitioning stops if
\begin{itemize}
\item All samples belong to same calss $ \rightarrow $ assign the
  class label to leaf
\item There are no attributes left $ \rightarrow $ majority voting to
  assign class label to leaf
\item There are no samples left
\end{itemize}

\subsection{Attribute selection}
At a given branch, the set of samples S to be classified has P
positive and N negative instances. \\
The entropy in S is \\
$ H(P, N) = - \frac{P}{P + N} log_2 \frac{P}{P + N} - \frac{N}{P + N}
log_2 \frac{N}{P + N} $ \\
Note
\begin{itemize}
\item If $ P = 0 $ or $ N = 0 $ \quad $ H(P, N) = 0 \rightarrow $ no
  doubt
\item If $ P = N $ \quad $ H(P, N) = 1 \rightarrow $ max doubt
\end{itemize}

\subsection{Information gain}
Attribute A partition S into $ S_1, S_2, \ldots, S_v $ \\
Entropy of A is \\
$ H(A) = \sum_{i = 1}^v \frac{P_i + N_i}{P + N} H(P_i, N_i) $ \\

The information gain obtained by splitting S using A is \\
$ Gain(A) = H(P, N) - H(A) $

\subsection{Pruning}
The construction phase does not filter out noise $ \rightarrow $
\textbf{overfitting}. \\
Pruning strategies
\begin{itemize}
\item Stop partitioning a node when large majority of sample is
  positive or negative
\item Build the full tree, then replace nodes with leaves labelled
  with the majority class, if accuracy does not change
\item Apply Minimum Description Length (MDL) principle
\end{itemize}

\subsection{MDL Pruning}
Let $ M_1, M_2, \ldots, M_n $ be a list of candidate models (i.e.\
trees). The best model is the one that minimizes \\
$ L(M) + L(D \mid M) $ \\
where
\begin{itemize}
\item $ L(M) $ is the length in bits of the description of the model
  (\#nodes, \#leaves, \#arcs, \ldots)
\item $ L(D \mid M) $ is the length in bits of the description of the
  data when encoded with the models (\#misclassifications)
\end{itemize}

\subsection{Extracting rules from trees}
Represent the knowledge in form of IF-THEN rules
\begin{itemize}
\item One rule is created for each path from the root to a leaf
\item Each attribute-value pari along a path forms a conjunction
\item The leaf node holds the class prediction
\end{itemize}
Rules are easier for humans to understand!

\subsection{Decision tree: continuous attributes}
With continuous attributes we cannot have a separate branch for each
values. Use \textbf{binary decision trees}.
\begin{itemize}
\item For continuous attributes A, a split is defined by $ val(A) < X
  $
\item For categorical attributes A, a split is defined by a subset $ X
  \subseteq domain(A) $
\end{itemize}

\subsubsection{Splitting continuous attributes}
\begin{itemize}
\item Sort the data according to attribute value
\item Determine the value of X which maximizes information gain by
  scanning data items
\end{itemize}
Only a relevant decision point if the class label changes

\subsection{Scalability of continuous attribute splits}
Naive
\begin{itemize}
\item At each step the data set is split in subsets that are
  associated with a tree node
\end{itemize}
Problem
\begin{itemize}
\item for evaluating which attribute to split data, needs to be sorted
  by these attributes
\item Becomes dominating cost
\end{itemize}

Presorting of data and preservation of order through
construction. Requires separate sorted attributes tables for each
attribute. \\
Updating attributes tables
\begin{itemize}
\item Attribute used for split: split table
\item Other attributes
  \begin{itemize}
  \item Build hash table associating tuple identifiers of data items
    with partition
  \item Select data from other attributes tables by scanning and
    probing the hash table
  \end{itemize}
\end{itemize}

\subsection{Characteristics of decision tree}
Strengths
\begin{itemize}
\item Automatic feature selection
\item Minimal data preparation
\item Non-linear
\item Easy to interpret and explain
\end{itemize}
Weaknesses
\begin{itemize}
\item Sensitive to small perturbations
\item Tend to overfit
\item Have to be re-train with new data
\end{itemize}

\subsection{Classification algorithms}
Alternatives
\begin{itemize}
\item Basic: Naïve bayes, kNN, logistic regression, \ldots
\item Ensemble: random forest, gradient boosting
\item Support vector machines
\item Neural networks: CNN, rNN, LSTM
\end{itemize}

\subsection{Ensemble methods}
\begin{itemize}
\item Take a collection of weak learners
\item Combine their results to make a single strong learners
\end{itemize}

Types
\begin{itemize}
\item \textbf{Bagging}: train learners in parrallel on different
  samples then combine outputs through voting or averaging
\item \textbf{Stacking}: combine model outputs using a second-stage
  learner like linear regression
\item \textbf{Boosting}: train learners on the filtered outputs of
  other learners
\end{itemize}

\subsection{Random forests}
Learn K different decision trees from independent samples. Vote
between different learners. Aggregate by majority vote.

\subsection{Why it works}
Assume 25 base classifiers with error rate $ \epsilon = 0.35 $ \\
Assume they are independent \\
Probability of wrong ensemble classification: \\
$ P(wrong) = \sum_{i=13} \binom{25}{i} \epsilon^i (1 - \epsilon)^{25 -
i} = 0.06 $

\subsection{Sampling strategies}

\begin{itemize}
\item Sampling data: select a subset of the data $ \rightarrow $ each
  tree is trained on different data
\item Sampling attributes: select a subset of attributes $ \rightarrow
  $ corresponding nodes in different trees (usually) don't use the
  same feature to split
\end{itemize}

\subsection{Random forest: algorithm}
\begin{itemize}
\item Draw k samples of size N from dataset
\item While constructing the decision tree, select a random set of
  \textbf{m attributes} out of the p attributes to infer split
  (feature bagging)
\end{itemize}
Typical parameters:
\begin{itemize}
\item $ m \approx \sqrt(p) $, or smaller
\item $ K \approx 500 $
\end{itemize}

\subsection{Characteristics of random forests}
Strengths
\begin{itemize}
\item Ensembles can model extremely complex decision boundaries
  without overfitting
\item Probably the most popular classifier for \textbf{dense data} ($
  \leq $ a few thousands features)
\item Easy to implement (train a lot of trees)
\item Parallelizes easily, good for map-reduce
\end{itemize}
Weaknesses
\begin{itemize}
\item Deep neural networks generally do better and sometimes even
  gradient boosted trees
\item Needs many passes over the data - at least the max depth of the
  trees
\item Relatively easy to overfit - hard to balance accuracy/fit
  tradeoff
\end{itemize}

%%% Local Variables:
%%% mode: latex
%%% TeX-master: "master"
%%% End:
